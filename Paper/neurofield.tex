\documentclass[preprint,review,10pt,authoryear,letterpaper]{elsarticle} 
\usepackage[textwidth=5cm]{todonotes} \oddsidemargin -0.5in
%\documentclass[final,authoryear,3p,times,twocolumn]{elsarticle}  \usepackage[disable]{todonotes}

%% Use the option review to obtain double line spacing
%% \documentclass[authoryear,preprint,review,12pt]{elsarticle}

%% Use the options 1p,twocolumn; 3p; 3p,twocolumn; 5p; or 5p,twocolumn
%% for a journal layout:
%% \documentclass[final,authoryear,1p,times]{elsarticle}
%% \documentclass[final,authoryear,1p,times,twocolumn]{elsarticle}
%% \documentclass[final,authoryear,3p,times]{elsarticle}

%% \documentclass[final,authoryear,5p,times]{elsarticle}
%% \documentclass[final,authoryear,5p,times,twocolumn]{elsarticle}

\usepackage{graphicx}
\usepackage[subrefformat=parens,labelformat=parens]{subfig}
\usepackage{amsmath,url}
\usepackage{amssymb}
\usepackage{booktabs} % Nicer tables
\usepackage{multirow} % Table cells spanning multiple rows
\usepackage{lineno}
\usepackage[noperiod]{jabbrv}

\biboptions{comma,sort&compress}

%% natbib.sty is loaded by default. However, natbib options can be
%% provided with \biboptions{...} command. Following options are
%% valid:

%%   round  -  round parentheses are used (default)
%%   square -  square brackets are used   [option]
%%   curly  -  curly braces are used      {option}
%%   angle  -  angle brackets are used    <option>
%%   semicolon  -  multiple citations separated by semi-colon (default)
%%   colon  - same as semicolon, an earlier confusion
%%   comma  -  separated by comma
%%   authoryear - selects author-year citations (default)
%%   numbers-  selects numerical citations
%%   super  -  numerical citations as superscripts
%%   sort   -  sorts multiple citations according to order in ref. list
%%   sort&compress   -  like sort, but also compresses numerical citations
%%   compress - compresses without sorting
%%   longnamesfirst  -  makes first citation full author list
%%
%% \biboptions{longnamesfirst,comma}

\biboptions{,comma,sort&compress}

\journal{Neuroscience Methods (any others?)}
  
  
\begin{document}

\begin{frontmatter}

%% Title, authors and addresses

%% use the tnoteref command within \title for footnotes;
%% use the tnotetext command for the associated footnote;
%% use the fnref command within \author or \address for footnotes;
%% use the fntext command for the associated footnote;
%% use the corref command within \author for corresponding author footnotes;
%% use the cortext command for the associated footnote;
%% use the ead command for the email address,
%% and the form \ead[url] for the home page:
%%
%% \title{Title\tnoteref{label1}}
%% \tnotetext[label1]{}
%% \author{Name\corref{cor1}\fnref{label2}}
%% \ead{email address}
%% \ead[url]{home page}
%% \fntext[label2]{}
%% \cortext[cor1]{}
%% \address{Address\fnref{label3}}
%% \fntext[label3]{}

\title{NeuroField}

%% use optional labels to link authors explicitly to addresses:
%% \author[label1,label2]{<author name>}
%% \address[label1]{<address>}
%% \address[label2]{<address>}

\author{P.K. Fung\corref{ffung}}
\ead{ffung@physics.usyd.edu.au}
\author{R.G. Abeysuriya\corref{}}

\author{P.A. Robinson\corref{}}
%\author{Anthony Krensel\corref{}}

\address{School of Physics, University of Sydney, New South Wales, Australia}
\cortext[ffung]{Corresponding author. Tel. +61 9036 7274}


\begin{abstract}
%% Text of abstract
Abstract

\end{abstract}

\begin{keyword}
%% keywords here, in the form: keyword \sep keyword
EEG \sep neurophysiology \sep methods \sep modeling
%% MSC codes here, in the form: \MSC code \sep code
%% or \MSC[2008] code \sep code (2000 is the default)

\end{keyword}

\end{frontmatter}

\linenumbers

%% main text
\section{Introduction}
\label{sec:introduction}


Motivation behind neural field theory: large-scale neural dynamics.

\begin{align*}
	D_{ab}V_{ab}(\mathbf{r},t) &= \nu_{ab}\phi_{ab}(\mathbf{r},t),\\
			Q_a(\mathbf{r},t) &= S_a \big[\sum_b V_{ab}(\mathbf{r},t) \big],\\
	\mathcal{D}_{ab}\phi_{ab}(\mathbf{r},t) &= Q_b(\mathbf{r},t-\tau_{ab}).
\end{align*}

NeuroField: a general code to solve the neural field theory by allowing users to:
\begin{enumerate}
	\item Specify an arbitrary number of populations and connections between populations;
	\item Specify the parameters for any objects, including populations, dendritic responses, firing responses, propagators, synapses, and stimulus pattern.
	\item Choose alternative wave propagation types, i.e. choose different forms of \(\mathcal{D}_{ab}\);
	\item Uses plastic synapses, i.e. \(\nu_{ab}=\nu_{ab}(\mathbf{r},t)\).
	\item Use different firing responses, i.e. change \(S_a\).
\end{enumerate}

\section{Method and Results}
\label{sec:theory}

NeuroField solves each equation within the Robinson et al. model with an object:
\begin{align*}
	P &= \nu_{ab}\phi_{ab}, & \mathtt{Couple}\\
	D_{ab}V_{ab} &= P, & \mathtt{Dendrite}\\
	Q_a &= S_a \big[\sum_b V_{ab} \big], & \mathtt{QResponse}\\
	\mathcal{D}_{ab}\phi_{ab} &= Q_b,&  \mathtt{Propag}
\end{align*}
with an arbitrary number of these objects, with each object may be a different type (e.g. constant synaptic coupling vs plastic synaptic coupling), and all parameter values may be tailored.

\begin{itemize}
\item Populations, can have as many as required, different customizable firing responses, bursting. Stimulus populations, different noise processes, pulsed stimulus, TMS
\item Propagators, wave propagator, spherical geometry
\item Couples, incorporate different types of plasticity
\item MATLAB helper scripts for visualization, power spectrum calculation, processing
\end{itemize}

Examples for single excitatory population, cortical population, corticothalamic model. Each example has a population diagram, and related results. All examples should preferably be published result?

\begin{itemize}
\item Plasticity results (Felix)
\item Corticothalamic model, compare analytic and neurofield result (Romesh)
\item Bursting populations (XL)
\item Seizures (XL, Romesh)
\end{itemize}

\section{Discussion}
\label{sec:discussion}

Overview of any tricky issues with the problem being solved (for example, the EEGLAB code paper mentions limitations of time/frequency decomposition). Discuss limitations or qualifiers on the usage of the code.

\begin{itemize}
\item Discussion regarding spatial components, grid size, noise amplitude with regard to approximations
\item CFL condition, automatically checked for. Also limitations on $\Delta x$ depending on $r_e$
\item Incorporating volume conduction
\end{itemize}
% \section{Conclusion}
% \label{sec:conclusion}
% Summary.

\section{Acknowledgements}
\label{sec:acknowledgements}
To add.

\section{References}
\bibliographystyle{elsarticle-harv}
\bibliography{neurofield}

\end{document}

%% References
%%
%% Following citation commands can be used in the body text:
%%
%%  \citet{key}  ==>>  Jones et al. (1990)
%%  \citep{key}  ==>>  (Jones et al., 1990)
%%
%% Multiple citations as normal:
%% \citep{key1,key2}         ==>> (Jones et al., 1990; Smith, 1989)
%%                            or  (Jones et al., 1990, 1991)
%%                            or  (Jones et al., 1990a,b)
%% \citep{key} is the equivalent of \citet{key} in author-year mode
%%
%% Full author lists may be forced with \citet* or \citep*, e.g.
%%   \citep*{key}            ==>> (Jones, Baker, and Williams, 1990)
%%
%% Optional notes as:
%%   \citep[chap. 2]{key}    ==>> (Jones et al., 1990, chap. 2)
%%   \citep[e.g.,][]{key}    ==>> (e.g., Jones et al., 1990)
%%   \citep[see][pg. 34]{key}==>> (see Jones et al., 1990, pg. 34)
%%  (Note: in standard LaTeX, only one note is allowed, after the ref.
%%   Here, one note is like the standard, two make pre- and post-notes.)
%%
%%   \citealt{key}          ==>> Jones et al. 1990
%%   \citealt*{key}         ==>> Jones, Baker, and Williams 1990
%%   \citealp{key}          ==>> Jones et al., 1990
%%   \citealp*{key}         ==>> Jones, Baker, and Williams, 1990
%%
%% Additional citation possibilities
%%   \citeauthor{key}       ==>> Jones et al.
%%   \citeauthor*{key}      ==>> Jones, Baker, and Williams
%%   \citeyear{key}         ==>> 1990
%%   \citeyearpar{key}      ==>> (1990)
%%   \citetext{priv. comm.} ==>> (priv. comm.)
%%   \citenum{key}          ==>> 11 [non-superscripted]
%% Note: full author lists depends on whether the bib style supports them;
%%       if not, the abbreviated list is printed even when full requested.
%%
%% For names like della Robbia at the start of a sentence, use
%%   \Citet{dRob98}         ==>> Della Robbia (1998)
%%   \Citep{dRob98}         ==>> (Della Robbia, 1998)
%%   \Citeauthor{dRob98}    ==>> Della Robbia


%% References with bibTeX database:

%% Authors are advised to submit their bibtex database files. They are
%% requested to list a bibtex style file in the manuscript if they do
%% not want to use elsarticle-harv.bst.

%% References without bibTeX database:

% \begin{thebibliography}{00}

%% \bibitem must have one of the following forms:
%%   \bibitem[Jones et al.(1990)]{key}...
%%   \bibitem[Jones et al.(1990)Jones, Baker, and Williams]{key}...
%%   \bibitem[Jones et al., 1990]{key}...
%%   \bibitem[\protect\citeauthoryear{Jones, Baker, and Williams}{Jones
%%       et al.}{1990}]{key}...
%%   \bibitem[\protect\citeauthoryear{Jones et al.}{1990}]{key}...
%%   \bibitem[\protect\astroncite{Jones et al.}{1990}]{key}...
%%   \bibitem[\protect\citename{Jones et al., }1990]{key}...
%%   \harvarditem[Jones et al.]{Jones, Baker, and Williams}{1990}{key}...
%%

% \bibitem[ ()]{}

% \end{thebibliography}

%%
%% End of file `elsarticle-template-harv.tex'.
